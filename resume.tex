%!TEX TS-program = xelatex
%!TEX encoding = UTF-8 Unicode
% Awesome CV LaTeX Template for CV/Resume
%
% This template has been downloaded from:
% https://github.com/posquit0/Awesome-CV
%
% Author:
% Claud D. Park <posquit0.bj@gmail.com>
% http://www.posquit0.com
%
% Template license:
% CC BY-SA 4.0 (https://creativecommons.org/licenses/by-sa/4.0/)
%


%-------------------------------------------------------------------------------
% CONFIGURATIONS
%-------------------------------------------------------------------------------
% A4 paper size by default, use 'letterpaper' for US letter
\documentclass[11pt, a4paper]{awesome-cv}

% Configure page margins with geometry
\geometry{left=1.4cm, top=.8cm, right=1.4cm, bottom=1.8cm, footskip=.5cm}

% Color for highlights
% Awesome Colors: awesome-emerald, awesome-skyblue, awesome-red, awesome-pink, awesome-orange
%                 awesome-nephritis, awesome-concrete, awesome-darknight
\colorlet{awesome}{awesome-orange}
% Uncomment if you would like to specify your own color
% \definecolor{awesome}{HTML}{3E6D9C}

% Colors for text
% Uncomment if you would like to specify your own color
% \definecolor{darktext}{HTML}{414141}
% \definecolor{text}{HTML}{333333}
% \definecolor{graytext}{HTML}{5D5D5D}
% \definecolor{lighttext}{HTML}{999999}
% \definecolor{sectiondivider}{HTML}{5D5D5D}

% Set false if you don't want to highlight section with awesome color
\setbool{acvSectionColorHighlight}{true}

% If you would like to change the social information separator from a pipe (|) to something else
\renewcommand{\acvHeaderSocialSep}{\quad\textbar\quad}

\newcommand*{\hcol}[1]{\hfill #1 \hspace{0.5cm}\null}
\usepackage{times,tcolorbox}
\newtcbox{\mybox}{colback=awesome-orange,boxrule=0pt,arc=6pt,
  boxsep=0pt,left=3pt,right=3pt,top=3pt,bottom=3pt, coltext=white, nobeforeafter}

%-------------------------------------------------------------------------------
%	PERSONAL INFORMATION
%	Comment any of the lines below if they are not required
%-------------------------------------------------------------------------------
% Available options: circle|rectangle,edge/noedge,left/right
% \photo[rectangle,edge,right]{./examples/profile}
\name{Philipp}{Jung}
\position{Senior Software Engineer {\enskip\cdotp\enskip} Research \& Development}\address{Karlsruher Str. 48, 69126, Heidelberg, GERMANY}

\mobile{(+49) 176 9787 9111}
\email{philipp.jung@mailbox.org}
%\dateofbirth{January 1st, 1970}
\homepage{philipp-jung.de}
\github{JungPhilipp}
\linkedin{philipp-jung95}
% \gitlab{gitlab-id}
% \stackoverflow{SO-id}{SO-name}
% \twitter{@twit}
% \skype{skype-id}
% \reddit{reddit-id}
% \medium{madium-id}
% \kaggle{kaggle-id}
% \hackerrank{hackerrank-id}
% \googlescholar{googlescholar-id}{name-to-display}
%% \firstname and \lastname will be used
% \googlescholar{googlescholar-id}{}
% \extrainfo{extra information}

\position{Software Engineer {\enskip\cdotp\enskip} Research \& Development}
\quote{``C++ and Rust enthusiast with 5+ years of professional and academic experience developing performance-critical C++ applications following DevOps and agile principles. Passionate about developer experience and looking to grow as an engineer and architect."}


%-------------------------------------------------------------------------------
\begin{document}

% Print the header with above personal information
% Give optional argument to change alignment(C: center, L: left, R: right)
\makecvheader[C]

% Print the footer with 3 arguments(<left>, <center>, <right>)
% Leave any of these blank if they are not needed
\makecvfooter
  {\today}
  {Philipp Jung~~~·~~~Curriculum Vitae}
  {\thepage}


\setlength{\textfloatsep}{0.1cm}
\addtolength{\parskip}{-0.1mm}

%-------------------------------------------------------------------------------
%	CV/RESUME CONTENT
%	Each section is imported separately, open each file in turn to modify content
%-------------------------------------------------------------------------------

\cvsection{Experience}
%\cvsubsection{Professional}
\vspace{-2.0mm}
\begin{cventries}
  \cventry
    {SAP SE}
    {Senior Software Engineer ``HANA Database Kernel: Storage Engine"}
    {Walldorf, DE}
    {01/2023 -- Present}
    {
      \begin{cvitems}
        %\item {TODO: Mergedog}
        \item {Reduced compile times by 10\%. Developed continuous monitoring to improve developer experience and reduce costs}
        \item {Part of an internal group to manage department-wide quality standards}
        %\item {Organized monthly newsletter to spread knowledge of modern/new C++ language and library features}
        %\item {Maintainer of CompileTimeExplorer an interactive tool to browse compile times of HANA by component, file, function and template}
        \item {Maintainer of CodeExplorer, a web application to interactively browse source code with compiler generated cross links and statistics}
        \item {\textbf{Technologies:} \textit{C++17/20/23, Rust, Python, Google Test/Benchmark Framework, CMake, Docker}}
      \end{cvitems}
    }
  \cventry
    {SAP SE}
    {Software Engineer ``HANA Database Kernel: Storage Engine"}
    {Walldorf, DE}
    {01/2021 -- 12/2022}
    {
      \begin{cvitems}
        \item {Optimized data structures for performance at scale (for very large table partitions of \textasciitilde 2 billion rows)}
        \item {Drove large refactoring efforts reducing code complexity and increase development velocity}
        \item {\textbf{Technologies:} \textit{C++17/20, Python, Google Test/Benchmark Framework, CMake, Docker}}
      \end{cvitems}
    }
  \cventry
    {Volume Graphics GmbH}
    {Software Engineer ``R\&D: Simulation on Computed Tomography Data"}
    {Heidelberg, DE}
    {04/2019 -- 12/2020}
    {
      \begin{cvitems}
        \item {Drove a system-wide architectural change improving startup times of a multi-million LOC cross-platform C++ application by 30\%}
        %\item {Achieved a paradigm shift to a trunk-based pull-request workflow with automated CI/CD pipeline based on Jenkins and Docker}
        %\item {Accomplished 90\% reduced feedback times from CI services as part of a two person task force using infrastructure-as-code principles}
        \item {Achieved 90\% reduction in CI feedback times by a paradigm shift toward a pull-request workflow with automated CI/CD pipeline}
        %\item {Developed voxel-based simulation techniques on large datasets from industrial computed tomography following agile principles}
        \item {\textbf{Technologies:} \textit{C++14/17, Python, Google Test Framework, CMake, QT, Conan, Docker, Jenkins, Scrum}}
      \end{cvitems}
    }
    \cventry
    {Institute of Computer Science }
    {Maintainer of libPRTL ``C++ library for Scalar/Vector and Tensor field processing"}
    {Heidelberg University, DE}
    {04/2017 -- 05/2019}
    {
      \begin{cvitems}
        \item {Provided reference implementations of widely used performance-critical algorithms in the field of scientific visualization}
        \item {Added rapid prototyping capabilities using python bindings (pybind11) for C++, significantly reducing development lead times}
        %\item {Added rapid prototyping capabilities using python bindings (pybind11) for C++}
        %\item {Widely used by PhD students and Master candidates of my former research group}
        %\item {Used clean-code principles and test-driven development with Catch2/Doctest including regular code reviews in a team of two}
        \item {\textbf{Technologies:} \textit{C++14/17, Python (pybind11), Eigen, Boost, Catch2/Doctest, CMake, CUDA, OpenMP, VTK, Gitlab CI, Docker}}
      \end{cvitems}
    }
    \cventry
    {Institute of Computer Science}
    {Research-Integrated Master }
    {Heidelberg University, DE}
    {04/2017 -- 03/2019}
    {
      \begin{cvitems}
        \item {Research project: ``Feature Extraction from Time-Dependent and Uncertain Vector Fields"}
        \item {Developed and implemented high-performance visualization algorithms for chaotic dynamical systems}
        %\item {Pushed test-driven-development to provide reference implementations and benchmarks for widely used algorithms}
        %\item {Established regular code reviews and more detailed git usage for student projects}
        \item {\textbf{Technologies:} \textit{C++14/17, Python, Eigen, CMake, CUDA, OpenMP, VTK}}
      \end{cvitems}
    }
    \cventry
    {Institute of Computer Science}
    {Student Research Assistant}
    {Heidelberg University, DE}
    {10/2016 -- 09/2017}
    {
      \begin{cvitems}
        \item {Designed and developed visualization tools for tropical cyclones with the ParaView/VTK framework using modern C++}
        %\item {Wrote and presented a paper to an international audience at a conference in Brazil}
        %\item {Established regular code reviews and more detailed git usage for student projects}
        \item {\textbf{Technologies:} \textit{C++11/14, Eigen, CMake, VTK}}
      \end{cvitems}
    }
\end{cventries}

\cvsection{Education}
\begin{cventries}
  \cventry
    {M.Sc. in Applied Computer Science, Minor in Mathematics {\enskip\cdotp\enskip} \textbf{German GPA: 1.0 with distinction}}
    {Heidelberg University}
    {Heidelberg, DE}
    {04/2017 -- 03/2019}
    {
      \begin{cvitems}
        \item {Thesis: ``On the Frame of Reference in Flow Visualization``, \textit{grade: 1.0 (A+)}}
        \item {\textbf{Technologies:} \textit{C++14/17, Eigen, Catch2/Doctest, CMake, Python, Matlab, Docker, VTK}}
      \end{cvitems}
    }
  \cventry
    {B.Sc. in Applied Computer Science, Minor in Economics {\enskip\cdotp\enskip} \textbf{German GPA: 1.5}}
    {Heidelberg University}
    {Heidelberg, DE}
    {10/2013 -- 03/2017}
    {
      \begin{cvitems}
        \item {Thesis: `Interpolation-Consistent Visualization of Bifurcations 2D Time-Dependent Vector Fields``, \textit{grade: 1.0 (A+)}}
        \item {\textbf{Technologies:} \textit{C++11, Eigen, Catch2/Doctest, CMake, VTK}}
      \end{cvitems}
    }
%   \shortentry
%     {M.Sc. in Applied Computer Science, Minor in Computational Lingustics; GPA: 4.0; with distinction}
%     {Heidelberg University}
%     {Heidelberg, DE}
%   \shortentry
%     {Heidelberg University}
%     {Heidelberg, DE}
\end{cventries}

\cvsection{Skills}
\begin{minipage}[t]{.35\linewidth}
{\vspace*{-0.45cm}\cvsubsection{Programming Languages }}
\vspace*{0.45cm}
\descriptionstyle{
     \begin{cvitems}
        \item {C++11/14/17/20/23 \hcol{Advanced}}
        \begin{cvitems_nested}
            \item {Google Test Framework \hcol{Advanced}}
            \item {(Modern) CMake \hcol{Advanced}}
            \item {Catch2/Doctest \hcol{Intermediate}}
            \item {Google Benchmark \hcol{Basic}}
        \end{cvitems_nested}
        \item {Rust \hcol{Intermediate}}
        \item {Python \hcol{Intermediate}}
        \item {SQL \hcol{Basic}}
        \item {JavaScript \hcol{Basic}}
      \end{cvitems}
}
\end{minipage}%
\begin{minipage}[t]{.35\linewidth}
{\vspace*{-0.45cm}\cvsubsection{Tools \& Frameworks}}
\vspace*{0.45cm}
\descriptionstyle{
      \begin{cvitems}
        \item {Git \hcol{Advanced}}
        \item {Gerrit \hcol{Advanced}}
        \item {Linux (Arch/Debian) \hcol{Advanced}}
        \item {Docker \hcol{Advanced}}
        \item {\LaTeX \hcol{Advanced}}
        \item {Agile(Scrum) \hcol{Intermediate}}
        \item {CI/CD(Jenkins/Github) \hcol{Intermediate}}
        \item {CUDA \hcol{Basic}}
        \item {Kubernetes \hcol{Basic}}
      \end{cvitems}
}
\end{minipage}%
\begin{minipage}[t]{0.3\linewidth}
{\vspace*{-0.45cm}\cvsubsection{Languages}}
\vspace*{0.45cm}
\descriptionstyle{
      \begin{cvitems}
        \item {English \hfill (TOEFL iBT 115/120) Fluent}
        \item {German \hfill Native}
      \end{cvitems}
}
\vspace*{.5cm}
\cvsubsection{Soft Skills} \vspace*{0.15cm}\\
\skilltypestylebox{
\mybox {Responsibility}
\mybox {Critical Thinking}\vspace*{0.2cm}
\mybox {Problem-Solving}
\mybox {Leadership}\vspace*{0.2cm}
\mybox {Teamwork}\hfill
\mybox {Effective Communication}
}
\end{minipage}

\cvsection{Personal Projects}
\begin{cventries}
  \cventry
    {Automated web-based visualization tool to track compilation times of (large) C++ projects}
    {Compile Time Explorer}
    {}
    {05/2023 -- Present}
    {
      \begin{cvitems}
        \item {Build on-top of clangs \textit{-ftime-trace} profile and \href{https://github.com/aras-p/ClangBuildAnalyzer}{ClangBuildAnalyzer}}
        \item {Allows interactive exploration of compile times by translation units, headers, functions and templates}
        \item {\textbf{Technologies:} \textit{Rust, JavaScript, Docker, NGINX}}
      \end{cvitems}
    }
  \cventry
    {Web-based utility cost calculation tool for apartment buildings and landlords}
    {Easy-Utility-Cost}
    {}
    {02/2019 -- Present}
    {
      \begin{cvitems}
        %\item {Web-based easy-to-use calculation tool for annual utility cost calculations for landlords}
        \item {Designed asynchronous business logic in Rust compiled to WebAssembly}
        \item {Developed a fast client-side page rendering based on JavaScript}
        %\item {Implemented server-client architecture with an NGINX-based web server running inside a Docker container}
        \item {\textbf{Technologies:} \textit{Rust, WebAssembly, JavaScript, Docker, NGINX}}
      \end{cvitems}
    }
\end{cventries}

\cvsection{Publications}
\begin{cventries}
  \cventry
    {M.~Würtz, D.~Aumiller, L.~Gundelwein, \textbf{P.~Jung} et al.}
    {DNA Accessibility of Chromatosomes Quantified by Automated Image Analysis}
    {Nature: Scientific Reports}
    {09/2019}
    {
      \begin{cvitems}
        \item {Interdisciplinary research project with the German Cancer Research Center (DKFZ)}
        \item {Designed an automated Matlab pipeline and implemented image denoising preprocessing steps using OpenCV}
        \item {\textbf{Technologies:} \textit{Matlab, C++, OpenCV, \LaTeX, Git}}
      \end{cvitems}
    }
  \cventry
    {\textbf{P.~Jung}, P.~Hausner, L.~Pilz, J.~Stern, C.~Euler, M.~Riemer, and F.~Sadlo}
    {Tumble-Vortex Core Line Extraction}
    {SIBGRAPI WVIS, 2017}
    {10/2017}
    {
      \begin{cvitems}
        %\item {Developed data analysis algorithms to extract features from vector fields}
        \item {Conceived and implemented a new algorithm in C++ to detect vortex core lines in previously unsolved cases}
        \item {Visualized analytical and simulated datasets using custom C++ plugins for ParaView/VTK}
        \item {\textbf{Technologies:} \textit{C++14/17, Eigen, Catch2/Doctest, CMake, ParaView, VTK}}
      \end{cvitems}
    }
\end{cventries}

\cvsection{Honors \& Awards}
\vspace*{-0.2cm}
\begin{cvhonors}
  \cvhonor
    {Research Scholarship}
    {sponsored by HGS MathComp and DFG}
    {}
    {08/2017 -- 07/2018}
  \cvhonor
    {Award for Exceptional Students}
    {sponsored by Beer Foundation}
    {}
    {10/2017}
  \cvhonor
    {Germany Scholarship}
    {sponsored by Leonie Wild Foundation}
    {}
    {10/2015 -- 09/2016}
\end{cvhonors}

\cvsection{Volunteer Activity}
\begin{cventries}
  \cventry
    {Student Representative}
    {Working Committee for Education}
    {Student Representation}
    {10/2016 -- 03/2019}
    {
      \begin{cvitems}
        \item {Organized weekly meetings coordinating university-wide student representation}
      \end{cvitems}
    }
  \cventry
    {Student Representative}
    {Senate Committee for Education}
    {Heidelberg University, DE}
    {12/2016 -- 09/2018}
    {
      \begin{cvitems}
        \item {Negotiated flexible university-wide attendance rules in lectures}
        \item {Represented student interest's concerning university-wide changes in study guidelines}
      \end{cvitems}
    }
  \cventry
    {Member}
    {Child Care Project e.V., Humanitarian Organization}
    {Heidelberg, DE}
    {06/2014 -- 02/2016}
    {
      \begin{cvitems}
        \item {Successfully organized  multiple fundraising events to finance a new primary school building in Uganda}
        \item {Designed and maintained website based on Hugo}
      \end{cvitems}
    }
\end{cventries}


%-------------------------------------------------------------------------------
\end{document}
